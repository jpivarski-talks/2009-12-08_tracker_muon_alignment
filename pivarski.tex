\documentclass[compress]{beamer}
\usepackage{ifthen,verbatim}

\newcommand{\isnote}{}
\xdefinecolor{lightyellow}{rgb}{1.,1.,0.25}
\xdefinecolor{darkblue}{rgb}{0.1,0.1,0.7}

%% Uncomment this to get annotations
%% \def\notes{\addtocounter{page}{-1}
%%            \renewcommand{\isnote}{*}
%% 	   \beamertemplateshadingbackground{lightyellow}{white}
%%            \begin{frame}
%%            \frametitle{Notes for the previous page (page \insertpagenumber)}
%%            \itemize}
%% \def\endnotes{\enditemize
%% 	      \end{frame}
%%               \beamertemplateshadingbackground{white}{white}
%%               \renewcommand{\isnote}{}}

%% Uncomment this to not get annotations
\def\notes{\comment}
\def\endnotes{\endcomment}

\setbeamertemplate{navigation symbols}{}
\setbeamertemplate{headline}{\mbox{ } \hfill
\begin{minipage}{5.5 cm}
\vspace{-0.75 cm} \small
\end{minipage} \hfill
\begin{minipage}{4.5 cm}
\vspace{-0.75 cm} \small
\begin{flushright}
\ifthenelse{\equal{\insertpagenumber}{1}}{}{Jim Pivarski \hspace{0.2 cm} \insertpagenumber\isnote/\pageref{numpages}}
\end{flushright}
\end{minipage}\mbox{\hspace{0.2 cm}}\includegraphics[height=1 cm]{../cmslogo} \hspace{0.1 cm} \includegraphics[height=1 cm]{../tamulogo} \hspace{0.01 cm} \vspace{-1.05 cm}}

\begin{document}
\begin{frame}
\vfill
\begin{center}
\textcolor{darkblue}{\Large Relative alignment of tracker and muon system \\ \vspace{0.2 cm} and track-based study of hardware results}

\vfill
\begin{columns}
\column{0.3\linewidth}
\begin{center}
\large
\textcolor{darkblue}{Jim Pivarski}
\end{center}
\end{columns}

\begin{columns}
\column{0.3\linewidth}
\begin{center}
\scriptsize
{\it Texas A\&M University}
\end{center}
\end{columns}

\vfill
 8 December, 2009

\end{center}
\end{frame}

%% \begin{notes}
%% \item This is the annotated version of my talk.
%% \item If you want the version that I am presenting, download the one
%% labeled ``slides'' on Indico (or just ignore these yellow pages).
%% \item The annotated version is provided for extra detail and a written
%% record of comments that I intend to make orally.
%% \item Yellow notes refer to the content on the {\it previous} page.
%% \item All other slides are identical for the two versions.
%% \end{notes}

\small

\begin{frame}
\frametitle{Outline}
\begin{itemize}\setlength{\itemsep}{0.75 cm}
\item Comparison of hardware geometry with track-based geometry
\item Studies of tracker shape using muon residuals
\end{itemize}
%% \hspace{-0.83 cm} \textcolor{darkblue}{\Large Outline2}
\end{frame}

\begin{frame}
\frametitle{Barrel hardware $-$ track-based}

\begin{columns}
\column{0.7\linewidth}
\includegraphics[height=\linewidth, angle=90]{NOV4DT_vs_HARDWARE_phi.pdf}

\includegraphics[width=\linewidth]{NovHardware_vs_phi.pdf}

\column{0.4\linewidth}
\begin{itemize}
\item Differences in chamber positions between hardware and track-based geometries
\only<1>{\item Top-left: reconstruction was following some LED reflections}
\only<1>{\item Bottom-left: corrected}
\only<2>{\item Sine curve: global position with respect to tracker
\begin{itemize}
\item $x \to 1.2$~mm
\item $y \to 4.5$~mm
\item $\phi_z \to 0.58$~mrad
\end{itemize}}
\end{itemize}

\only<1>{\includegraphics[width=\linewidth]{hwbarrel_spots.png}}
\only<2>{\includegraphics[width=\linewidth]{adjustment_cartoon.pdf}}
\end{columns}
\end{frame}

\begin{frame}
\frametitle{After global translation/rotation}

\includegraphics[width=\linewidth]{NovHardware_vs_tracks_as_histograms.pdf}

\begin{itemize}
\item RMS of $x'$ deviations (top-left) is only 1.8~mm after removing global offset in transverse plane

\item Similar to photogrammetry, which had an RMS of 1.6~mm
\end{itemize}
\end{frame}

\begin{frame}
\frametitle{Other translational d.o.f.}

\begin{columns}
\column{0.7\linewidth}
\includegraphics[width=\linewidth]{NovHardwareADJUSTED_vs_y.pdf}

\includegraphics[width=\linewidth]{NovHardwareADJUSTED_vs_z.pdf}

\column{0.4\linewidth}
\begin{itemize}
\item Top: $y'$ differences (direction parallel to beamline)

\item Bottom: $z'$ differences (radial)

\item These are more consistent within sector groups

\item $y'$ has a clear trend with respect to wheel
\end{itemize}
\end{columns}
\end{frame}

\begin{frame}
\frametitle{Map plots}

\vspace{-0.25 cm}
\begin{center}
\includegraphics[width=0.8\linewidth]{DTvsphi_st1whC_y.png}
\end{center}

\vspace{-0.5 cm}
\begin{itemize}
\item Map plots show that the remaining differences are not due to tracker distortions
\item For example, the statement ``relative misalignment of sectors 4 and 5 is 6~mm'' is tracker-independent and propagation-independent
\item When the hardware geometry correctly describes the muon system as a rigid body, the differences with respect to track-based will be a {\it smooth} function in these plots (e.g. between sectors 11 and 12)
\end{itemize}

\mbox{\hspace{-0.5 cm} \scriptsize For all plots, see \textcolor{blue}{https://hypernews.cern.ch/HyperNews/CMS/get/muon-alignment/423/1.html}\hspace{-1 cm}}
\end{frame}

\begin{frame}
\frametitle{Resolving tracker distortions}

\begin{itemize}
\item We can identify tracker global distortions using muon chamber data
\item Without introducing circularity when we later align the muon system with tracker tracks
\end{itemize}

\vspace{0.1 cm}
\begin{columns}
\column{0.5\linewidth}
\includegraphics[width=\linewidth]{method.pdf}

\column{0.5\linewidth}
\begin{itemize}
\item Independence from muon alignment: plot residuals in {\it only one muon layer} \\ (wh~0, st~1, sec~10, lay~2) \\ and disregard global position of that layer

\item Look at muon residuals as a function of $p_T$

\item We only assume that the muon layer is in one location that can't be a function of the $p_T$ of the tracks used to measure it
\end{itemize}
\end{columns}

\end{frame}

\begin{frame}
\frametitle{$p_T$-dependent muon residuals}

\begin{columns}
\column{0.57\linewidth}

\includegraphics[width=\linewidth]{residuals_real.pdf}

\includegraphics[width=\linewidth]{curvature_real.pdf}

\includegraphics[width=\linewidth]{momenta_real.pdf}

\column{0.5\linewidth}
\begin{itemize}
\item Three ways of looking at it:
\begin{itemize}
\item as a muon residual ($\Delta x$)
\item tracker curvature error ($\Delta \kappa = \Delta x \frac{d\kappa}{dx}$, $\kappa = q/p_T$)
\item tracker momentum error ($\Delta p_T = \Delta x \frac{dp_T}{dx}$)
\end{itemize}
where $\frac{d\kappa}{dx}$ and $\frac{dp_T}{dx}$ come from track propagator, numerically

\item $\frac{d\kappa}{dx}$ is nearly constant for a single DT layer (depends on distance from tracker)

\item From muon hits, we learn something which is purely about the tracker's shape

\item If errors were from $\vec{B}(\vec{x})$ or $dE/dx$, top plot would be antisymmetric, not symmetric
\end{itemize}
\end{columns}
\end{frame}

\begin{frame}
\frametitle{Ambiguity from muon alignment}

\begin{columns}
\column{0.57\linewidth}

\includegraphics[width=\linewidth]{residuals_realshift.pdf}

\includegraphics[width=\linewidth]{curvature_realshift.pdf}

\includegraphics[width=\linewidth]{momenta_realshift.pdf}

\column{0.5\linewidth}
\begin{itemize}
\item To avoid a circular argument, we should keep the position of the muon chamber as a free parameter in this study

\item Knowledge of tracker shape will later be used to determine positions of muon chambers (track-based alignment)

\item Freedom to make low-$p_T$ region ``correct'' and high-$p_T$ region ``wrong''

\item Still, difference in curvature between low- and high-$p_T$ regions is the same: this should constrain models of the tracker's shape
\end{itemize}
\end{columns}
\end{frame}

\begin{frame}
\frametitle{On one slide}

\begin{itemize}
\item Freedom in DT layer position yields different fractional $p_T$ error profiles
\item One thing which is not consistent with the data is: zero momentum bias across the whole $p_T$ range
\item Minbias alignment, with primary vertex constraint, information
  from resonance masses, etc.\ can further correct or constrain this
\end{itemize}

\vspace{0.25 cm}
\includegraphics[width=0.5\linewidth]{residuals_real_both.pdf}\mbox{\hspace{0.2 cm}}\includegraphics[width=0.5\linewidth]{momenta_real_both.pdf}
\end{frame}

\begin{frame}
\frametitle{Does this analysis make sense?}

\begin{columns}
\column{0.57\linewidth}

\includegraphics[width=\linewidth]{residuals_ideal.pdf}

\includegraphics[width=\linewidth]{curvature_ideal.pdf}

\includegraphics[width=\linewidth]{momenta_ideal.pdf}

\column{0.5\linewidth}
\begin{itemize}
\item When applied to ideal MC, everything is perfect

\item That's good (not a software problem or anything)
\end{itemize}
\end{columns}
\end{frame}

\begin{frame}
\frametitle{Straw-man global distortions (1)}

\begin{columns}
\column{0.57\linewidth}

\includegraphics[width=\linewidth]{residuals_layerRotation.pdf}

\includegraphics[width=\linewidth]{curvature_layerRotation.pdf}

\includegraphics[width=\linewidth]{momenta_layerRotation.pdf}

\column{0.5\linewidth}
\includegraphics[width=\linewidth]{layerRotation.png}

\begin{itemize}
\item $r\phi$ rotation of tracker layers as a function of $r$

\item Tracker track $\chi^2$ is highly sensitive to this, so it has been ruled out

\item Nevertheless, it would produce a similar effect
\end{itemize}
\end{columns}
\end{frame}

\begin{frame}
\frametitle{Straw-man global distortions (2)}

\begin{columns}
\column{0.57\linewidth}

\includegraphics[width=\linewidth]{residuals_sagitta.pdf}

\includegraphics[width=\linewidth]{curvature_sagitta.pdf}

\includegraphics[width=\linewidth]{momenta_sagitta.pdf}

\column{0.5\linewidth}
\includegraphics[width=\linewidth]{sagitta.png}

\begin{itemize}
\item $r\phi$ rotation of tracker layers as a function of $\phi$

\item Cosmic ray tracker tracks are not very sensitive to this

\item It also produces a similar effect

\item That doesn't mean that it's the only explanation
\end{itemize}
\end{columns}
\end{frame}

\begin{frame}
\frametitle{Why it's not sagitta}

\begin{itemize}
\item Plot tracker curvature error $\Delta (q/p_T)$ on the color scale (GeV$^{-1}$)
\item Horizontal axes as indicated
\end{itemize}

\includegraphics[width=0.5\linewidth]{2d_real.pdf}
\includegraphics[width=0.5\linewidth]{2d_sagitta.pdf}

\begin{itemize}
\item Real distortion is more a function of $q/p_T$ than $\phi$
\item Sagitta error is more a function of $\phi$ than $q/p_T$
\end{itemize}

\end{frame}

\begin{frame}
\frametitle{Trying other distortions}

\begin{itemize}
\item Following an idea by Markus, trying $f(x, y) = (x + 5\times10^{-6} y^2, y)$
\item Quick toy MC: causes $5\times10^{-6}$~cm$^{-1}$ = 0.0005~GeV$^{-1}$ distortions
\begin{itemize}
\item without large residuals
\item without affecting cosmic ray splitting
\item unfortunately, also no dependence on $p_T$ (this isn't the one)
\item collisions and resonances may be sensitive to it
\end{itemize}
\item Should try this and variants on it with the real tracker \mbox{alignment tools\hspace{-1 cm}}
\end{itemize}

\begin{center}
\begin{minipage}{0.8\linewidth}
\includegraphics[width=0.55\linewidth]{toymc_tracker.pdf}
\includegraphics[width=0.4\linewidth]{toymc_results.pdf}
\end{minipage}
\end{center}
\end{frame}

%% \section*{First section}
%% \begin{frame}
%% \begin{center}
%% \Huge \textcolor{blue}{First section}
%% \end{center}
%% \end{frame}

\begin{frame}
\frametitle{Conclusions}
\begin{itemize}\setlength{\itemsep}{0.5 cm}
\item Barrel hardware alignment produced, actively being compared with track-based data

\item Muon residuals provide some constraints on the global tracker shape

\vspace{-0.25 cm}
\begin{itemize}\setlength{\itemsep}{0.1 cm}
\item including estimates of momentum bias
\item should be part of a larger program of integrating knowledge from cosmics, minbias, resonances, etc.
\item large project: looking for people interested in working on it
\item tentative interest: Markus Stoye, Roberto Castello
\end{itemize}
\end{itemize}
\label{numpages}
\end{frame}

\end{document}
